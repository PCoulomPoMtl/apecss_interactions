\chapter{Using APECSS}

\section{Installation}
\label{sec:installation}

Installing APECSS is easy. After downloading APECSS in the folder {\tt <path to APECSS>}, define the following environment variables:\vspace{-1em}
\begin{itemize}[noitemsep]
\item {\tt APECSS\_DIR} to the folder in which APECSS is located. Using {\tt bash}, for instance, simply execute the command {\tt export APECSS\_DIR=<path to APECSS>} or, even better, add this command to your {\tt bash} profile.
\item {\tt USRLIB\_DIR} to the folder in which {\tt libm.a} or {\tt libm.dylib} (the standard \textit{math} library) is located. This may, for instance, be {\tt /usr/lib64/} on Linux systems or {\tt /usr/lib/} on MacOS systems.
\end{itemize}

Now, navigate into the folder {\tt \$APECSS\_DIR/lib} and execute {\tt ./compile\_lib.sh}. This shell script will compile the APECSS library using {\tt cmake} with the {\tt CMakeLists.txt} file provided in this folder. By default, APECSS is compiled with double precision and in \textit{Release} mode, meaning all optimization flags are enabled. That's it, you've successfully installed APECSS!

If you wish to install APECSS with \textit{Debug} flags, simply change the line
\begin{lstlisting}[style=CStyle,numbers=none]
  cmake CMakeLists.txt -DCMAKE_BUILD_TYPE=Release
\end{lstlisting}\vspace{-0.75em}
to
\begin{lstlisting}[style=CStyle,numbers=none]
  cmake CMakeLists.txt -DCMAKE_BUILD_TYPE=Debug
\end{lstlisting}\vspace{-0.75em}
in the {\tt ./compile\_lib.sh} shell script and, if applicable, in the shell script that compiles the desired example.

\section{Running APECSS}

There are several ways in which you can use the APECSS library. You can either incorporate selected features of APECSS into your own software code or you can program an interface to use APECSS as a standalone software tool. 

\subsection{The {\tt *.apecss} options file}

The {\tt *.apecss} file is the primary way of passing options, such as the size of the bubble, the density of the liquid or the type of results you want to have written out, to APECSS. 

Any {\tt *.apecss} file may contain the following sections (each terminated with {\tt END}):
\vspace{-1em}
\begin{itemize}[noitemsep]
  \item {\tt BUBBLE}: Information related to the bubble, such as its initial radius $R_0$ or the Rayleigh-Plesset model that is used to solve its dynamics.
  \item {\tt GAS}: Properties and equation of state of the gas.
  \item {\tt LIQUID}: Properties, type (i.e.~Newtonian or viscoelastic) and equation of state of the liquid.
  \item {\tt INTERFACE}: Properties of the gas-liquid interface. 
  \item {\tt RESULTS}: Results of the bubble dynamics and the acoustic emissions that should be written out.
  \item {\tt ODESOLVER}: Parameters of the ODE solver.
\end{itemize}
The options file is read by the functions {\tt apecss\_*\_readoptions()}. Any of the sections in the {\tt *.apecss} file, and any of the options that may be defined within each section, are optional; they are read if they are present, otherwise the default values (typically set in {\tt apecss\_*\_setdefaultoptions()}) or values set in the code calling it will be used. Comments can be added using {\tt \#}.

The relevant options that are available are discussed in the following chapters of this documentation in the context of the theoretical framework of APECSS. For instance, the available options used to define a certain Rayleigh-Plesset model are discussed in Section \ref{sec:rpmodels}, where the theory of the implemented Rayleigh-Plesset models is described. 

\subsection{Examples}

Some representative examples are available in the {\tt \$APECSS\_DIR/examples} folder. Each directory contains the following:\vspace{-1em}
\begin{itemize}[noitemsep]
  \item A {\tt README.md} file explaining the purpose and specificities of this/these example(s).
  \item A {\tt src/} folder with a file called {\tt *\_apecss.c} that acts as the standalone interface to the APECSS library. This file contains the {\tt main()} function and any additional functionality required to simulate a specific scenario.
  \item A {\tt build/} folder containing the {\tt CMakeLists.txt} file and a shell script {\tt compile.sh} with which this example can be compiled using the command {\tt ./compile.sh}.
  \item One or several {\tt *.apecss} file(s) in which the options for a specific case are defined.
\end{itemize}

In the examples provided in the {\tt \$APECSS\_DIR/examples} folder, the name of the  {\tt *.apecss} file is passed as an argument with the call to run APECSS, e.g.~executing {\tt ./<APECSS-example> -options run.apecss} to use an options file named {\tt run.apecss}.

Detailed information about each example, how to run it and what the results might be compared to can be found in the accompanying {\tt README.md} file.

\section{Programming in and with APECSS}

All functions are located in source files ({\tt *.c}) that relate to parts of the code, distinguished by physical phenomena (e.g.~{\tt emissions.c}), fluid type (e.g.~{\tt liquid.c}) or computational operations (e.g.~{\tt results.c}). 

All declarations and definitions are located in the header file {\tt include/apecss.h}. 

\subsection{Macros}

Macros are used as shortcuts to define frequently-used constants (e.g.~{\tt APECSS\_PI}), for frequently-used computational operations (e.g.~{\tt APECSS\_MAX}) and for computational operations that depend on the chosen machine precision (e.g.~{\tt APECSS\_SQRT}). Furthermore, options related to different numerical models are represented by logically named flags.

\subsubsection{Macros related to machine precision}

APECSS can be used with different floating point precisions: double precision (default) and long double precision ({\tt APECSS\_PRECISION\_LONGDOUBLE}).

Based on the chosen precision, {\tt APECSS\_FLOAT} is defined as the standard floating point type. In addition, the following precision-dependent computational operations are defined based on the chosen floating point precision:\vspace{-1em}
\begin{itemize}[noitemsep]
  \item {\tt APECSS\_ABS(a)}: Absolute value of $a$.
  \item {\tt APECSS\_CEIL(a)}: $a$ rounded to the nearest integer larger than $a$.
  \item {\tt APECSS\_COS(a)}: Cosine of $a$.
  \item {\tt APECCS\_EPS}: Returns a value that is close to machine epsilon.
  \item {\tt APECSS\_EXP(a)}: $\text{e}$ to the power $a$.
  \item {\tt APECSS\_LOG(a)}: Natural logarithm of $a$.
  \item {\tt APECSS\_POW(a,b)}: $a$ to the power $b$.
  \item {\tt APECSS\_SIN(a)}: Sine of $a$.
  \item {\tt APECSS\_SMALL}: Returns a number significantly smaller than machine epsilon.
  \item {\tt APECSS\_SQRT(a)}: Square root of $a$.
  \item {\tt APECSS\_STRINGTOFLOAT(a)}: Conversion of string $a$ to a float.
\end{itemize}
To ensure compatibility for different floating point precisions, it is paramount to use the standard floating point type {\tt APECSS\_FLOAT} and the operator definitions given above consistently throughout APECSS.

\subsubsection{Computational operations and predefined constants}

Macros that provide a shortcut to frequently-used computational operations are:\vspace{-1em}
\begin{itemize}[noitemsep]
  \item {\tt APECSS\_POW2(a)}: Returns $a^2$.
  \item {\tt APECSS\_POW3(a)}: Returns $a^3$.
  \item {\tt APECSS\_POW4(a)}: Returns $a^4$.
  \item {\tt APECCS\_MAX(a,b)}: Returns the maximum of $a$ and $b$.
  \item {\tt APECSS\_MAX3(a,b,c)}: Returns the maximum of $a$, $b$ and $c$.
  \item {\tt APECSS\_MIN(a,b)}: Returns the minimum of $a$ and $b$.
  \item {\tt APECSS\_MIN3(a,b,c)}: Returns the minimum of $a$, $b$ and $c$.
\end{itemize}

Macros that provide a shortcut to frequently-used constants are:\vspace{-1em}
\begin{itemize}[noitemsep]
  \item {\tt APECSS\_PI}: Returns $\pi=3.1415926535897932384626433832795028841972$.
  \item {\tt APECSS\_E}: Returns $\mathrm{e}=2.7182818284590452353602874713526624977572$.
  \item {\tt APECSS\_ONETHIRD}: Returns $1/3$.
  \item {\tt APECSS\_ONESIXTH}: Returns $1/6$.
  \item {\tt APECCS\_AVOGADRO}: Returns the Avogadro constant, $N_\mathrm{A}=6.02214076 \times 10^{23} \, \mathrm{mol}^{-1}$.
  \item {\tt APECSS\_LN\_OF\_2}: Returns the natural logarithm of 2, $\ln(2)=0.693147180559945309$.
  \item {\tt APECSS\_LN\_OF\_10}: Returns the natural logarithm of 10, $\ln(10)=2.302585092994045684$.
  \item {\tt APECSS\_LARGE}: Returns a large number, defined as $10^{15}$.
\end{itemize}

\subsubsection{Flags for model options}

All model options are represented by human-readable flags. 

If explicitly indicated as such, these flags are defined in such a way (with integer values being a multiple of 2), that a bit-wise comparison can be performed. Bit-wise comparison may be used for options that are checked frequently and for options that can have several building blocks.

Options of the embedded Runge-Kutta scheme of \citet{Dormand1980} used to discretize the governing ODEs:\vspace{-1em}
\begin{itemize}[noitemsep]
  \item {\tt APECSS\_RK54\_7M}: RK5(4)7M (minimum truncation) coefficients of \citet{Dormand1980}
  \item {\tt APECSS\_RK54\_7S}: RK5(4)7S (stability optimized) coefficients of \citet{Dormand1980}
\end{itemize}

Rayleigh-Plesset schemes:\vspace{-1em}
\begin{itemize}[noitemsep]
  \item {\tt APECSS\_BUBBLEMODEL\_RP}: Standard Rayleigh-Plesset model, Eq.~\eqref{eq:standardRP}.
  \item {\tt APECSS\_BUBBLEMODEL\_RP\_ACOUSTICRADIATION}: Rayleigh-Plesset model with acoustic radiation damping, Eq.~\eqref{eq:modRP}.
  \item {\tt APECSS\_BUBBLEMODEL\_KELLERMIKSIS}: Keller-Miksis model, Eq.~\eqref{eq:keller}.
  \item {\tt APECSS\_BUBBLEMODEL\_GILMORE}: Gilmore model, Eq.~\eqref{eq:gilmore}.
\end{itemize}

Equation of state of the gas:\vspace{-1em}
\begin{itemize}[noitemsep]
  \item {\tt APECSS\_GAS\_IG}: Ideal gas EoS.
  \item {\tt APECSS\_GAS\_HC}: Ideal gas EoS with van-der-Waals hardcore.
  \item {\tt APECSS\_GAS\_NASG}: Noble-Abel-stiffened-gas EoS.
\end{itemize}

Equation of state of the liquid:\vspace{-1em}
\begin{itemize}[noitemsep]
  \item {\tt APECSS\_LIQUID\_TAIT}: Tait EoS.
  \item {\tt APECSS\_LIQUID\_NASG}: Noble-Abel-stiffened-gas EoS.
\end{itemize}

Viscoelasticity of the liquid:\vspace{-1em}
\begin{itemize}[noitemsep]
  \item {\tt APECSS\_LIQUID\_NEWTONIAN}: Newtonian liquid.
  \item {\tt APECSS\_LIQUID\_KELVINVOIGT}: Kelvin-Voigt solid.
  \item {\tt APECSS\_LIQUID\_ZENER}: Zener solid (standard linear solid model).
  \item {\tt APECSS\_LIQUID\_OLDROYDB}: Oldroyd-B liquid.
\end{itemize}

Lipid monolayer coating of the gas-liquid interface (bit-wise):\vspace{-1em}
\begin{itemize}[noitemsep]
  \item {\tt APECSS\_LIPIDCOATING\_NONE}: No lipid monolayer coating.
  \item {\tt APECSS\_LIPIDCOATING\_MARMOTTANT}: Lipid monolayer coating described by the model of \citet{Marmottant2005}.
  \item {\tt APECSS\_LIPIDCOATING\_GOMPERTZFUNCTION}: Redefine the Marmottant model with a Gompertz function \citep{Guemmer2021}.
\end{itemize}

Acoustic excitation applied to the bubble:\vspace{-1em}
\begin{itemize}[noitemsep]
  \item {\tt APECSS\_EXCITATION\_NONE}: No external excitation. 
  \item {\tt APECSS\_EXCITATION\_SIN}: Sinusoidal excitation.
\end{itemize}

Model to compute the acoustic emissions of the bubble (bit-wise):\vspace{-1em}
\begin{itemize}[noitemsep]
  \item {\tt APECSS\_EMISSION\_NONE}: Emissions are not modelled.
  \item {\tt APECSS\_EMISSION\_INCOMPRESSIBLE}: Emissions are assumed to occur in an incompressible fluid.
  \item {\tt APECSS\_EMISSION\_FINITE\_TIME\_INCOMPRESSIBLE}: Emissions are assumed to occur in an incompressible fluid, but the finite propagation speed given by the speed of sound is taken into account.
  \item {\tt APECSS\_EMISSION\_QUASIACOUSTIC}: Emissions are modelled under the quasi-acoustic assumption of \citet{Trilling1952} and \citet{Gilmore1952}.
  \item {\tt APECSS\_EMISSION\_KIRKWOODBETHE}: A model based on the Kirkwood-Bethe hypothesis (EKB, GFC, HPE) is used.
  \item {\tt APECSS\_EMISSION\_EV}: Emissions are modelled using the explicit expression based on the Kirkwood-Bethe hypothesis.
  \item {\tt APECSS\_EMISSION\_SIV}: Emissions are modelled using the spatially-integrated velocity, based on the Kirkwood-Bethe hypothesis, of \citet{Gilmore1952}.
  \item {\tt APECSS\_EMISSION\_TIV}: Emissions are modelled using the temporally-integrated velocity, based on the Kirkwood-Bethe hypothesis, of \citet{Hickling1963}.
\end{itemize}

\subsubsection{Others}
\label{sec:using_macros_others}

Macros for the output of certain results produced by APECSS:\vspace{-1em}
\begin{itemize}[noitemsep]
  \item {\tt APECSS\_RESULTS\_DISCARD}: Discard the results and do not write results to file.
  \item {\tt APECSS\_RESULTS\_WRITE}:  Write all data to file in one go.
  \item {\tt APECSS\_RESULTS\_APPEND}: Append the results file with a new batch of results.
\end{itemize}

Other predefined macros are used to define the length of strings and arrays, as well as to help with debugging:\vspace{-1em}
\begin{itemize}[noitemsep]
  \item {\tt APECSS\_DATA\_ALLOC\_INCREMENT}: The increment for dynamic re-allocation of arrays.
  \item {\tt APECSS\_STRINGLENGTH}: The standard length of a string.
  \item {\tt APECSS\_STRINGLENGTH\_SPRINTF}: The standard length of a string to be written out in the terminal.
  \item {\tt APECSS\_STRINGLENGTH\_SPRINTF\_LONG}: The standard length of a long string to be  written out in the terminal.
  \item {\tt APECSS\_WHERE}: Outputs in the terminal the file name and line number where the macro is called.
  \item {\tt APECSS\_WHERE\_INT(a)}: Outputs in the terminal the file name and line number where the macro is called, plus the integer value $a$.
  \item {\tt APECSS\_WHERE\_FLOAT(a)}: Outputs in the terminal the file name and line number where the macro is called, plus the floating point value $a$.
\end{itemize}


\subsection{Structures}

Structures ({\tt struct}) are used in APECSS to group variables and functions, and to provide a modular layout of the code that enables reusing different parts of it.

The structure {\tt APECSS\_Bubble} is the central structure of APECSS as it contains all the information related to a bubble. There is, of course, no {\it a priori} limit on how many copies of this structure a simulation can have, for instance, a multi-bubble simulation with 100 bubbles would naturally have 100 objects of type {\tt struct APECSS\_Bubble}. Aside from key information about the bubble, such as the bubble radius, the {\tt APECSS\_Bubble} structure contains pointers to the properties of the liquid the bubble is immersed in ({\tt struct APECSS\_Liquid}), the properties of the gas the bubble contains ({\tt struct APECSS\_Gas}) as wel as the properties of its interface ({\tt struct APECSS\_Interface}). If applicable, the {\tt APECSS\_Bubble} structure also points to the structure with the information of the driving acoustic excitation ({\tt struct APECSS\_Excitation}), the structure handling the acoustic emissions ({\tt struct APECSS\_Emissions}), and the structure containing the desired results ({\tt struct APECSS\_Results}). 

The (optional) structure {\tt APECSS\_Emissions} is, as the name suggests, related to the acoustic emissions of a bubble. If allocated, it contains information about how to handle the acoustic emissions, function pointers referring to the functions used to advance the acoustic emissions using a Lagrangian wave tracking approach and, very importantly, the linked list of emissions nodes ({\tt struct APECSS\_EmissionNode}) that carry the actual information of the acoustic emissions. The structure {\tt APECSS\_EmissionNode}  holds the information (e.g.~radial position, velocity, enthalpy, pressure) associated with a specific emission node.

The (optional) structure {\tt APECSS\_Results} holds all the results the user may want to have written out. For performance reasons, the results are in general not written to disk on-the-fly, but are stored in arrays and dumped to disk at the end of the simulation. The {\tt APECSS\_Results} structure contains optional structures for the results of the Rayleigh-Plesset model ({\tt struct APECSS\_ResultsBubble}) and for the acoustic emissions ({\tt APECSS\_ResultsEmissions}).

\subsection{Important functions}

Whenever using APECSS, at least one {\tt APECSS\_Bubble} structure has to be available and, if multiple bubbles are part of a simulation, each bubble has to be represented by a separate {\tt APECSS\_Bubble} structure. A single {\tt APECSS\_Bubble} structure is allocated and initialized as follows:
\begin{lstlisting}[style=CStyle,numbers=none]
  struct APECSS_Bubble *Bubble = (struct APECSS_Bubble *) malloc(sizeof(struct APECSS_Bubble));
  apecss_bubble_initializestruct(Bubble);
\end{lstlisting}\vspace{-0.75em}
The function {\tt apecss\_bubble\_initializestruct()} ensures that all pointers (to arrays, other structures and functions) that are part of the {\tt APECSS\_Bubble} structure are initialized to {\tt NULL}. This is important because the processing of options and any checks for allocation depend on {\tt NULL} to indicate that a given pointer in not yet allocated or not in use.
Then, we wish to set default values for the bubble parameters and read the relevant options from the options file:
\begin{lstlisting}[style=CStyle,numbers=none]
  apecss_bubble_setdefaultoptions(Bubble);
  apecss_bubble_readoptions(Bubble, OptionsDir);
\end{lstlisting}\vspace{-0.75em}
While setting default values is strongly advised, it is optional. The relative path to the options file may be set as
\begin{lstlisting}[style=CStyle,numbers=none]
char OptionsDir[APECSS_STRINGLENGTH];
sprintf(OptionsDir, "./run.apecss"); // Relative path to the options file.
\end{lstlisting}

In general, the properties of a gas ({\tt struct APECSS\_Gas}), a liquid ({\tt struct APECSS\_Liquid}) and a gas-liquid interface ({\tt struct APECSS\_Interface}), as well as the parameters for the ODE solver ({\tt struct APECSS\_NumericsODE}), have to be associated with a bubble, through the structure pointers {\tt *Gas}, {\tt *Liquid}, {\tt *Interface} and {\tt *NumericsODE} readily available in the {\tt APECSS\_Bubble} structure. In a single-bubble simulation this is obviously straightforward, we have one gas, one liquid and one interface that are associated with the bubble. In addition we have one set of solver parameters. In a multi-bubble simulation, however, we likely also have, for instance, only a single liquid (i.e.~all bubbles are situated in the same body of liquid), to which, in this case, all bubbles are associated to. Generally, the user is free in defining as many gases, liquids, interfaces and sets of solver parameters as deemed necessary, the only rule is that each bubble has to be associated with a gas, a liquid, an interface and a single set of solver parameters. Once these structures holding the fluid properties and the solver parameters are allocated, default values ought to be set, the user-defined options are read from file and the bubble(s) is/are successfully associated with its/their desired fluid properties and solver parameters. For a single-bubble simulation, this may look in a general form like:
\begin{lstlisting}[style=CStyle,numbers=none]
  struct APECSS_Gas *Gas = (struct APECSS_Gas *) malloc(sizeof(struct APECSS_Gas));
  struct APECSS_Liquid *Liquid = (struct APECSS_Liquid *) malloc(sizeof(struct APECSS_Liquid));
  struct APECSS_Interface *Interface = (struct APECSS_Interface *) malloc(sizeof(struct APECSS_Interface));
  struct APECSS_NumericsODE *NumericsODE = (struct APECSS_NumericsODE *) malloc(sizeof(struct APECSS_NumericsODE));

  apecss_gas_setdefaultoptions(Gas);
  apecss_liquid_setdefaultoptions(Liquid);
  apecss_interface_setdefaultoptions(Interface);
  apecss_odesolver_setdefaultoptions(NumericsODE);

  apecss_gas_readoptions(Gas, OptionsDir);
  apecss_liquid_readoptions(Liquid, OptionsDir);
  apecss_interface_readoptions(Interface, OptionsDir);
  apecss_odesolver_readoptions(NumericsODE, OptionsDir);

  Bubble->Gas = Gas;
  Bubble->Liquid = Liquid;
  Bubble->Interface = Interface;
  Bubble->NumericsODE = NumericsODE;
\end{lstlisting}\vspace{-0.75em}
Note that opening and reading the options file from disk is a relatively expensive (we are talking about a few microseconds) operation. For some of the single-bubble examples in {\tt \$APECSS\_DIR/examples} reading the options file is the most expensive operation by some margin. Therefore, if performance is of the essence, it might be worth hard coding the options instead of reading the options file.

After reading the options file, the {\tt apecss\_*\_processoptions()} functions are called to process options and make the relevant modeling choices:
\begin{lstlisting}[style=CStyle,numbers=none]
  apecss_gas_processoptions(Gas);
  apecss_liquid_processoptions(Liquid);
  apecss_interface_processoptions(Interface);
  apecss_odesolver_processoptions(NumericsODE);
  apecss_bubble_processoptions(Bubble);
\end{lstlisting}\vspace{-0.75em}
Processing the given options \uline{correctly} is critical to the working of APECSS.

Now that all options have been read and processed, the bubble has to be initialized based on the given options. This includes, for instance, computing the initial gas pressure inside the bubble (if it is not specified by the user) or, if applicable, the hardcore radius of the bubble.
\begin{lstlisting}[style=CStyle,numbers=none]
  apecss_bubble_initialize(Bubble);
\end{lstlisting}

The heart of APECSS is, of course, the solver for the bubble dynamics. The solver is split into three separate functions that (i) initialize the solver, (ii) run the solver and (iii) wrap up (i.e.~finalize) the solver:
\begin{lstlisting}[style=CStyle,numbers=none]
  apecss_bubble_solver_initialize(Bubble);
  apecss_bubble_solver_run(tend, Bubble);
  apecss_bubble_solver_finalize(Bubble);
\end{lstlisting}\vspace{-0.75em}
The initialization of the solver with {\tt apecss\_bubble\_solver\_initialize()} makes sure all counters, the solution error variable and, if applicable, result variables are initialized correctly. The function {\tt apecss\_bubble\_solver\_run()} contains the time-stepping procedure that executes the solver until a specified end time {\tt tend}, given as the first argument of the function call. In the examples found in {\tt \$APECSS\_DIR/examples}, the function {\tt apecss\_bubble\_solver\_run()} is only called once with {\tt tend = Bubble->tEnd}, i.e.~the end of the simulation. However, the user is free to call the function {\tt apecss\_bubble\_solver\_run()} an \uline{arbitrary number of times}, with any meaningful end time. This facilitates coupling APECSS to other numerical software frameworks, where {\tt tend} then could for instance be the end of the next time-step of a fluid dynamics solver. As an example simply chopping the simulation up into five equal-sized parts would look like:
\begin{lstlisting}[style=CStyle,numbers=none]
  apecss_bubble_solver_initialize(Bubble);
  apecss_bubble_solver_run(0.2 * tend, Bubble);
  apecss_bubble_solver_run(0.4 * tend, Bubble);
  apecss_bubble_solver_run(0.6 * tend, Bubble);
  apecss_bubble_solver_run(0.8 * tend, Bubble);
  apecss_bubble_solver_run(tend, Bubble);
  apecss_bubble_solver_finalize(Bubble);
\end{lstlisting}\vspace{-0.75em}
A solver run is ended with the function {\tt apecss\_bubble\_solver\_finalize()} where, for instance, the arrays and linked list of the acoustic emissions are freed.

\subsection{The void data pointer}

Additional data associated with a bubble, an emission node, a gas, a liquid or an interface might be needed for more complex simulations, for instance neighbor information when multiple bubbles interact with each other or the thermal conductivity and heat capacity of the gas inside the bubble when heat transfer is considered. In APECSS, to retain flexibility and avoid unnecessary overhead, the structures {\tt struct APECSS\_Bubble}, {\tt struct APECSS\_EmissionNode}, {\tt struct APECSS\_Gas}, {\tt struct APECSS\_Liquid} and {\tt struct APECSS\_Interface} contain a \texttt{void} pointer called \texttt{user\_data} specifically for the purpose of associating additional data with those structures. This void pointer is not associated with a specific data type, but rather points to some memory location, i.e.~a memory address. 

A typical use of this void pointer would be to generate a structure that contains all additional data and point the void pointer to the address of this structure. For instance, the void pointer of the bubble structure is used in such a way in the example {\tt \$APECSS\_DIR/examples/laserinducedcavitation}. The structure
\begin{lstlisting}[style=CStyle,numbers=none]
  struct LIC
  {
    APECSS_FLOAT tauL, Rnbd, Rnc1, Rnc2, tmax1, tmax2;
  };
\end{lstlisting}\vspace{-0.75em}
is allocated
\begin{lstlisting}[style=CStyle,numbers=none]
  struct LIC *lic_data = (struct LIC *) malloc(sizeof(struct LIC));
\end{lstlisting}\vspace{-0.75em}
and associated with the void pointer
\begin{lstlisting}[style=CStyle,numbers=none]
  Bubble->user_data = lic_data;
\end{lstlisting}\vspace{-0.75em}
The data stored in this structure can then be used or read by again assigning the correct data type
\begin{lstlisting}[style=CStyle,numbers=none]
  struct LIC *lic_data = Bubble->user_data;
\end{lstlisting}\vspace{-0.75em}
and used as
\begin{lstlisting}[style=CStyle,numbers=none]
  tmax = lic_data->tmax1;
\end{lstlisting}\vspace{-0.75em}
At the end of the simulation, when the data is no longer needed, the memory occupied by the structure is freed
\begin{lstlisting}[style=CStyle,numbers=none]
  free(lic_data);
\end{lstlisting}%\vspace{-0.75em}

The example {\tt \$APECSS\_DIR/examples/gastemperature} uses both the void pointers associated with the bubble and the gas.

\subsection{A word on function pointers}

APECSS uses function pointers extensively. Function pointers are an elegant means in {\tt C} to add complexity and functionality yet still retain a slim code, avoid redundant code and, if nothing else, avoid a large number of costly conditional statements. However, function pointers can quickly make a code unreadable and obfuscate what is actually happening, if they are used without care. In order to keep the use of function pointers in APECSS transparent, the adopted convention is that \uline{all} function pointers are set in {\tt apecss\_*\_processoptions()} functions, e.g.~{\tt apecss\_gas\_processoptions()}.

\subsection{Code formatting}
\label{sec:clang}

To ensure a consistent formatting, please use a \textit{clang} formatter that formats the file automatically upon saving. The file defining the formatting of the APECSS source code ({\tt .clang-format}) is part of the repository. A \textit{clang} formatter (supported by most IDEs and editors) should be used for contributions to APECSS. The formatter should recognize this {\tt .clang-format} file automatically.

\section{Output}
By default, APECSS does \uline{not} write out any output, neither to the terminal nor to a file. 

Basic information about the version and compile options of APECSS can be written out in the terminal using the function {\tt apecss\_infoscreen()} and any user-defined message by passing the desired string to the function {\tt apecss\_writeonscreen()}, as demonstrated in many of the available examples.

Various simulation results associated with the bubble dynamics and the acoustic emissions can be written to file by passing the appropriate options to APECSS (see Sections \ref{sec:bubble_results} and \ref{sec:emissions_results} for more details). In the interest of performance (writing data to file is very expensive), the results are stored in arrays and only written to file when the appropriate {\tt apecss\_results\_*\_write()} function is called, at the end of the simulation or at any appropriate point during the simulation, as shown in the examples. Exporting the results associated with the radial bubble dynamics using the function {\tt apecss\_results\_rayleighplesset\_write()} and the recording of the acoustic emissions at predefined spatial locations using the function {\tt apecss\_results\_emissionsspace\_write()} additionally requires the user to tell APECSS how the data should be handled, using the macros described in Section \ref{sec:using_macros_others}. Either the appropriate file is appended ({\tt APECSS\_RESULTS\_APPEND}), which makes sense if the results should be written out multiple times during a simulation, or the data is written to file in one go ({\tt APECSS\_RESULTS\_WRITE}), which should be used if the results are written out once at the end of the simulation. In both cases, the file is created automatically if it does not exist yet. Alternatively, the option {\tt APECSS\_RESULTS\_DISCARD} deletes the results without writing them to file, which can be useful if the results are used to collect, for instance, some statistics during the simulations but should not be stored in a file.

\section{Units}

APECSS assumes SI units or any appropriate combination of SI units at all times, e.g.~when reading user-defined options, in all internal computations and when outputting data. To avoid any misunderstanding, the SI base units are the following:\vspace{-1em}
\begin{itemize}[noitemsep]
  \item Time in seconds [s]
  \item Length in meter [m]
  \item Mass in kilogram [kg]
  \item Temperature in Kelvin [K]
  \item Electric current in Ampere [A] 
  \item Amount of substance in mole [mol]
  \item Luminosity in candela [cd]
\end{itemize}

